\documentclass[12pt]{article}
\usepackage{geometry}
\usepackage{amsmath}
\usepackage{amsthm}
\usepackage{amssymb}
\usepackage{mathrsfs}
\usepackage{parskip}
\usepackage{enumerate}
\usepackage{stmaryrd}
\usepackage{listings}
\usepackage{fullpage}
\usepackage{qtree}

\begin{document}

\title{CS 350 Notes}
\author{Matthew Visser}
\date{Oct  4, 2011}
\maketitle

\section{Assignment Issues}

\subsection{Errors in A1}

If you detect an error, you can
\begin{itemize}
	\item assert
	\item panic
	\item fail silently
	\item put return values in the lock struct.
\end{itemize}

In other settings you might be able to change return values, but since the API
is define dfor us, we can't.

\subsection{\texttt{sy2} Test}

Will run the lock test. This test passing once does not guarantee that your lock
implementation is correct.

\section{System Calls}

There are 3 ways to enter the kernel:
\begin{itemize}
	\item syscalls
	\item interrupts
	\item exceptions
\end{itemize}

On MIPS, these all jump to \texttt{0x80000080}.

\subsection{The Exception Handler}

Every thread has two stacks, one in user mode and one in kernel mode. The thread
library points to the kernel stack.

The exception handler does the following:
\begin{enumerate}
	\item allocates a trap fram on thread's kernel stack and saves user-level
		application's state.
	\item calls \texttt{mips\_trap}
	\item when this returns, restores application and processor state
	\item issues MIPS \texttt{jr} and \texttt{rfe} instructions to return
		control to application.
\end{enumerate}

The calling code can either call an exception, interrupt, or syscall.

\Tree
[
.{\texttt{0x80000080}}
[
.{exception (asm)}
[
.{\texttt{common\_exception} (asm)}
[
.{\texttt{mips\_trap} (C)}
\texttt{mips\_syscall()}
\texttt{mips\_interrupt()}
\texttt{mips\_exception()}
]]]]

A syscall returns an error code and an optional return value. Pass a pointer to
the return value variable to the syscall functions to be able to change it. This
is OK because nothing other than \texttt{mips\_syscall()} will be calling it.

\section{Exceptions and Interrupts}

\subsection{Exceptions}

\begin{itemize}
	\item Conditions that occur during execution of user code
	\item Detected by hardware.
	\item hardware calls exception handler
	\item examples are
		\begin{itemize}
			\item page faults
			\item divide by zero
		\end{itemize}
\end{itemize}

\subsection{Alignment}

When we load or store, we need to load or store from an alignment that's
divisible by the size of the register.  Unaligned memory causes exceptions.

\subsection{Interrupts}

\begin{itemize}
	\item A third mechanism to enter kernel code
	\item similar to exceptions, but caused by hardware devices, not program
		execution:
		\begin{itemize}
			\item Network interface may generate an interrupt
			\item hard drive
		\end{itemize}
\end{itemize}

\end{document}
% vim: tw=80
