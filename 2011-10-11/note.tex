\documentclass[12pt]{article}
\usepackage{geometry}
\usepackage{amsmath}
\usepackage{amsthm}
\usepackage{amssymb}
\usepackage{mathrsfs}
\usepackage{parskip}
\usepackage{enumerate}
\usepackage{stmaryrd}
\usepackage{listings}
\usepackage{fullpage}

\begin{document}

\title{CS 350 Notes}
\author{Matthew Visser}
\date{Oct 11, 2011}
\maketitle

\section{Paging}

Page size $p$, virtual address $v$. Page sizes are always a power of two.

Page number:
\begin{equation}
	\left \lfloor \frac{v}{p} \right \rfloor
\end{equation}

This can be represented by \texttt{v >> y}.

Offset is
\begin{equation}
	v \pmod p
\end{equation}

\subsection{A Page Size Example}
We have a 32-bit address, and the offset is 10 bits.  How many bits do we have
for the page number? What's the page size? How many pages do we have in an
address space?

\begin{enumerate}
	\item We have 22 bits for the page number.
	\item Page size is $2^10 = 1\text{KB}$ bytes.
	\item Number of pages are $2^22 = 4\text{M}$ pages. 
\end{enumerate}

There are some advantages to have small pages
\begin{itemize}
	\item Reduce fragmentation
	\item more efficient memory usage
\end{itemize}
but there are some disadvantages
\begin{itemize}
	\item large page table.
	\item processes have many pages, always doing lookups.
\end{itemize}
Typical page sizes are 256 bytes --- 8 kB.

\section{Using Page Tables}

The kernel writes the address of the page table for a process in the \emph{page table
base register}.  When the MMU wants to use address lookup it 
\begin{itemize}
	\item looks in the page table base register for the page table,
	\item divide the address into a page number and offset,
	\item perform lookup on the page,
	\item and then construct the physical address.
\end{itemize}
See diagram on slide 7 of \texttt{vm.pdf}.

There are some things in place to protext memory:
\begin{itemize}
	\item The kernel can set the \emph{valid bit} of addresses set to 0.
	\item The MMU can enforce a read only bit as well, so a program can't write
		to memory that is read only.
	\item Many MMU's have a bit called the \emph{no-execute bit}.
\end{itemize}

What the kernel does with the page table:
\begin{itemize}
	\item Saves the state of the page table on every context switch.
	\item create and manage the page tables
	\item manage physical memory
	\item handle MMU errors
\end{itemize}

The MMU needs to
\begin{itemize}
	\item translate addresses.
	\item check for and raise exceptions
\end{itemize}

The array with an entry for each frame is called the \emph{core map}.

\subsection{Midterm Problem on Address Translation}

From the winter 2008 midterm.

\textbf{Page table of process}: $p$.

The process has 9 pages numbered 0-8.

\begin{center}
	\begin{tabular}{|c|r|}
		\hline
		Page Number & Physical Address \\
		\hline
		0 & 0x8d10\\
		1 & 0x1004\\
		2 & 0x004a\\
		3 & 0x5500\\
		4 & 0x2220\\
		5 & 0x2221\\
		6 & 0x2222\\
		7 & 0x222a\\
		8 & 0x5558\\
		\hline
	\end{tabular}
\end{center}

Doing this translation with a 4 bit offset, we get the following.
\begin{center}
	\begin{tabular}{|c|c|}
		\hline
		Virtual Addresses & Physical Address\\
		\hline
		0x00003a8 & 0x5500a8 \\
		0x0001004 & NO TRANSLATION \\
		0x0000022 & 0x8d1022 \\
		\hline
	\end{tabular}
\end{center}

Doing this with a 12-bit offset, we get:
\begin{center}
	\begin{tabular}{|c|c|}
		\hline
		Virtual Addresses & Physical Address\\
		\hline
		0x00003a8 & 0x8d103a8 \\
		0x0001004 & 0x1004004 \\
		0x0000022 & 0x8d10022 \\
		\hline
	\end{tabular}
\end{center}

\subsection{Remaining Issues}

We have a few issues
\begin{itemize}
	\item address translation speed
	\item sparseness
	\item where do we put the kernel in this picture?
\end{itemize}

\subsubsection{Speed of Address Translation}

\begin{itemize}
	\item Execution of each machine instruction may involve one or more
		translations.
		\begin{itemize}
			\item one for fetching an instruction
			\item One or more for instruction operands
		\end{itemize}
	\item Every translation is a memory lookup, which is slow.
	\item We can get around this using a \emph{translation look-aside buffer
		(TLB)} in the MMU.
\end{itemize}

\subsubsection{The TLB}
\begin{itemize}
	\item Each entry contains a (page number, frame number) pair.
	\item if the address translation is cached in the TLB, then the page
		table is avoided.
	\item If the hardware controls this TLB, then the hardware does the address
		translation using it (unless there is a page fault).
	\item If this is a software TLB, then the kernel has to keep this page table
		cached.
	\item Some MMUs load the TLB themselves, others raise an exception if a page
		number isn't int he TLB, and the kernel must load the translation to it.
		\begin{itemize}
			\item MIPS has a software controlled TLB.
		\end{itemize}
	\item The hardware dictates the format of the TLB. The kernel gets to
		control how the TLB is used and what is loaded to it in a software
		controlled TLB.
	\item TLB is much faster because it's a simple hardware lookup.
	\item If we switch processes, there will be a lot of exceptions thrown
		because old mappings will be populated.
		\begin{itemize}
			\item We either accept the faults, or
			\item We flush the TLB every time we have a context switch.
		\end{itemize}
\end{itemize}

\subsubsection{The MIPS TLB}

\dots or not. Next time!

\end{document}
% vim: tw=80
