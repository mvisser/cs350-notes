\documentclass[12pt]{article}
\usepackage{geometry}
\usepackage{amsmath}
\usepackage{amsthm}
\usepackage{amssymb}
\usepackage{mathrsfs}
\usepackage{parskip}
\usepackage{enumerate}
\usepackage{stmaryrd}
\usepackage{listings}
\usepackage{fullpage}

\begin{document}

\title{CS 350 Notes}
\author{Matthew Visser}
\date{Oct 13, 2011}
\maketitle

\section{Memory Management}
\subsection{TLB}

The TLB works because we have
\begin{itemize}
	\item Large pages
	\item Locality
		\begin{itemize}
			\item spacial --- memory is in regions. If you're accessing one
				region, you're probably going to be accessing regions next to
				it.
			\item temporal
		\end{itemize}
\end{itemize}

We will formalize the notion of locality later.

What does an entry in the MIPS TLB look like?

The MIPS page size is 4kB = $2^{12}$ bytes $\implies$ offset is $12$ bits.

Have 
\begin{itemize}
	\item first word
		\begin{itemize}
			\item 20 bits Virtual Page Number (VPN)
			\item 6 bits ASID --- address space ID
			\item 6 bits empty
		\end{itemize}
	\item Second Word
		\begin{itemize}
			\item 20 bits PFN
			\item 1 bit N --- not cacheable
			\item 1 bit D --- dirty
				\begin{itemize}
					\item changes to 1 when written to.
					\item if 0, can be trusted
				\end{itemize}
			\item 1 bit V --- Valid bit
			\item 1 bit G --- Global bit. Tells the MMU to ignore the ASID when
				1.
			\item 8 bits null.
		\end{itemize}
\end{itemize}

We can modify the TLB with the functions
\begin{itemize}
	\item \texttt{TLB\_Write}
	\item \texttt{TLB\_Random}
	\item \texttt{TLB\_Read}
	\item \texttt{TLB\_Probe}
\end{itemize}

\subsection{Segmentation}

Different parts of the program are segments. The different segments are loaded
into different parts of physical memory. The segments are described in the
\emph{segment table}. This has information on the physical address of the
segment. See slide 18 of \texttt{vm.pdf}.

The segment table points to the page table when we want to combine them. See
slide 20 for a diagram.

The segmentation addresses the problem of sparse addresses.

We don't, however, have segmentation in OS/161.

\subsection{Address Translation}

\begin{itemize}
	\item The MMU translates addresses
	\item If there is an error, we have a TLB fault (called an \emph{address
		exception}).
	\item The \texttt{vm\_fault} function handles this exception for the kernel.
		It uses information fromt he current process' \texttt{addrspace} to
		construct and load a TLB entry for the page.
	\item On return from fault handling, the processor re-tries to do the
		previous action.
\end{itemize}

Virtual addresses must start at an address that is divisible by the page size.
In our representation, last three digits of hex must be 0.

\end{document}
% vim: tw=80
