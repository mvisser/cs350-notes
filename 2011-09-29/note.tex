\documentclass[12pt]{article}
\usepackage{geometry}
\usepackage{amsmath}
\usepackage{amsthm}
\usepackage{amssymb}
\usepackage{mathrsfs}
\usepackage{parskip}
\usepackage{enumerate}
\usepackage{stmaryrd}
\usepackage{listings}
\usepackage{fullpage}

\begin{document}

\title{CS 350 Notes}
\author{Matthew Visser}
\date{Sep 29, 2011}
\maketitle

\section{Deadlocks}

Use ``no hold and wait'' and ``resource ordering'' in OS/161.

\subsection{Deadlock Detection and Recovery}

Determine if there is a deadlock and terminate a thread.  The thread you
terminate doesn't matter, it's a choice that's up to the scheduler. Ideally
choose one that makes sense with regard to resources, but there are trade-offs.

We can decide if the system is deadlocked by looking at a resource allocation
graph. We want an algorithm that looks at these graphs and tries to make the
same decisions about them. We can define state by
\begin{itemize}
    \item $D_i$ demand vector for thread $T_i$.
    \item $A_i$ current allocation for thread $T_i$.
    \item $U$ unallocated resources
\end{itemize}
We also have $R$ for a scratch resource and $f_i$ a boolean that determines if
$T_i$ can finish.

See the algorithm on slide 38 of \texttt{synch.pdf}.

\section{C Programming Tips}

\subsection{Modular Code}

The OS/161 source tree is separated into different modules with
\texttt{.h} (headers with function prototypes and type declarations) and
\texttt{.c} files (with function definitions and global variables.)

These modules will be using and calling each other. When one module uses
another, all it cares about is the header file and what is defined there. For
example, using \texttt{synch.h} with \texttt{\#include "synch.h"}, will give you
all of the functions in that header file.

\subsection{Arrays and Pointers}

Consider the code:
\begin{verbatim}
int a[5];
int x;
int *p;
\end{verbatim}

The definition of \texttt{a} is an array that holds 5 integers on the stack.
The variable \texttt{a} is actually just a pointer to the first address in the
array, so \texttt{*a} is equivalent to \texttt{a[0]}. \texttt{*(a+1)} is
equivalent to \texttt{a[1]}.

A pointer can be set tothe value of another pointer, such as \texttt{p = a}, or
to the address of a variable such as \texttt{p = \&x}. It can also be equal to
an allocated array pointer by doing
\begin{verbatim}
p = (int *) malloc( 16 * sizeof(int));
\end{verbatim}

In OS/161 we will use \texttt{kmalloc} instead of \texttt{malloc}.  We will also
use \texttt{kprintf} instead of \texttt{printf}.

\subsection{The \texttt{const} Keyword}

We can declare a variable that should never change like this
\begin{verbatim}
const int x = 5;
const int *x;
int * const x;
\end{verbatim}
The first is a constant number. The second is a pointer to a constant integer.
The third is a pointer that is constant, that points to an integer value. A
trick to understand this is to read from right to left.

\subsection{Other Notes}

\begin{itemize}
    \item Don't access pointers after they have been freed
    \item Don't return a pointer to a variable on the stack, \textit{i.e.}, a
        variable in the argument list or declared in the function without
        \texttt{kmalloc}.
\end{itemize}

\section{Processes}

A \textbf{process} is an abstraction of a program in execution. It contains
\begin{itemize}
    \item An address space
    \item one or more thread of execution
    \item Other resources such as files, sockets, attributes, \textit{etc.}
\end{itemize}

A process is a user level program, not something in the kernel.

\subsection{Multiprogramming}

\begin{itemize}
    \item Means having multiple processes at the same time
    \item Most modern computers do this
    \item All processes share the resources of the machine.  The kernel needs to
        co-ordinate these resources.
    \item Processes \emph{timeshare} the processor. This is controlled by the
        operating system.
    \item The OS ensures that processes are isolated from each other.
        Interprocess communication should be possible, but only when explicitly
        requested by the processes.
\end{itemize}

\section{The Kernel}

\begin{itemize}
    \item The kernel is a program with code and data.
    \item The kernel runs in a privileged execution mode, while other programs
        don't. This is a special mode offered by the CPU.
    \item The hardware has to switch between privileged execution mode and user
        mode.
    \item What does privileged mode mean? What can we do with it?
        \begin{itemize}
            \item control hardware: disable interrupts, halt the machine,\dots
            \item protect and isolate itself (the kernel) from processes.
        \end{itemize}
    \item privileges vary from platform.
\end{itemize}

\section{System Calls}

\begin{itemize}
    \item Interface between processes and the kernel. They allow a process to
        tell the kernel they want a resource or to do something with hardware.
    \item The process can use this to do things like
        \begin{itemize}
            \item Open a file: \texttt{opn}
            \item create a pipe: \texttt{write}
            \item Create another process: \texttt{fork}
            \item increase address space: \texttt{brk}
        \end{itemize}
    \item This is done by calling a special instruction (\texttt{syscall} in
        MIPS) This jumps to a specific address in the kernel. This address in
        MIPS is \texttt{0x80000080}.
    \item We don't want user code to jump in arbitrary addresses in the kernel
        code, but this is prevented by the fixed address.
\end{itemize}






\end{document}
% vim: tw=80
