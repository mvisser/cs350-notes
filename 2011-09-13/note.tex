\documentclass[12pt]{article}
\usepackage{geometry}
\usepackage{amsmath}
\usepackage{amsthm}
\usepackage{amssymb}
\usepackage{mathrsfs}
\usepackage{parskip}
\usepackage{enumerate}
\usepackage{stmaryrd}
\usepackage{listings}
\usepackage{fullpage}
\usepackage{hyperref}

\begin{document}

\title{CS 350 Notes}
\author{Matthew Visser}
\date{Sep 13, 2011}
\maketitle

\section*{Views of an OS}

\subsection{Intro}

\begin{itemize}
    \item Web page: \url{http://www.student.cs.uwaterloo.ca/~cs350}
    \item discussions on \textit{Piazza}
    \item textbook
        \begin{itemize}
            \item \textit{Operating System Concepts} by Silberschatz, Galvin and
                Gagne (8th ed.)
            \item not relying in lecture, but important resource.
        \end{itemize}
    \item course notes: printouts of slides
\end{itemize}

\subsection{System View}

\begin{itemize}
    \item manage hardware resources
    \item allocate resources
    \item use resources it shares
\end{itemize}

\subsection{Implementation View}

\begin{itemize}
    \item concurrent - needs to be because it manages other programs
    \item hardware interactions impose time constraints
\end{itemize}

\subsection*{Terminology}

\begin{description}
    \item[kernel] \hfill\\
        Part of the OS that responds to system calls, interupts, etc.
    \item[operating system] \hfill\\
        The whole includes kernel and may include other related programs that
        provide services e.g. ls, cd.
\end{description}


\end{document}
% vim: tw=80
