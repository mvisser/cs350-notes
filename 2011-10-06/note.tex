\documentclass[12pt]{article}
\usepackage{geometry}
\usepackage{amsmath}
\usepackage{amsthm}
\usepackage{amssymb}
\usepackage{mathrsfs}
\usepackage{parskip}
\usepackage{enumerate}
\usepackage{stmaryrd}
\usepackage{listings}

\begin{document}

\title{CS 350 Notes}
\author{Matthew Visser}
\date{Oct  6, 2011}
\maketitle

\section{Processes}

The kernel keeps track of who made a process, resources it holds, address space,
\textit{etc}. This is stored in a struct called the \emph{process control
block}.

\texttt{struct thread} contains a lot of information about a running process as
well, but the subprocess that is a thread.

A process with one thread is called a \emph{sequential process}. A process with
more than one thread is called a \emph{concurrent process}.

\subsection{Timesharing}

Context switching happens inside the kernel. When a thread enters the kernel,
this is an opportunity to do a context switch.

The role of the timer interrupt is to make sure that a thread periodically
enters the kernel. Pre-emption can still happen any time a thread enters the
kernel.


\subsection{System Calls for Process Management}

See slide 35 of \texttt{proc.pdf}.

\begin{description}
	\item[\texttt{fork}] creates a clone of the calling process. Fork returns
		from both the calling process, \textit{i.e.}, the original one, and the
		child process, \textit{i.e.}, the new one. When it returns, if it is in
		the child process, it returns 0, if it is the parent, it returns the
		process ID of the new process.
	\item[\texttt{execv}] replaces the address space of the calling process with
		an address space loaded from an executable file.
	\item[\texttt{waitpid}] allows a process to wait for another process to
		finish.
	\item[\texttt{getpid}] gets the process id
	\item[\texttt{getusage}] tells you how many resources you are using and how
		much.
\end{description}


There is a distinction between \emph{kernel threads} and \emph{user
threads}.  Kernel threads are dispatched by the kernel. User threads are
possibly not. User threads do not get pre-empted because they are in user-space,
so they need to yield to each other. The reason for having these user threads is
because kernel threads are fairly heavy-weight and user threads are lighter.
Sometimes, operating systems don't support multiple threads per user process as
well.  There are some threads used by the kernel for kernel ``housekeeping'' only.

\section{Memory}

\subsection{Virtual vs Physical Addresses}

Memory has both physical and virtual addresses.
\begin{itemize}
	\item physical addresses
		\begin{itemize}
			\item Physical addresses are provided by the machine.  There is only
				one physical address per machine.
			\item The size of the physical address determines the amount of
				addressable memory.
			\item MIPS has 32-bits, so we can only have $2^{32} \text{bytes} = 4
				\text{GB}$ of physical memory.
		\end{itemize}
	\item Virtual addresses
		\begin{itemize}
			\item There is one virtual address space \emph{per process}.
		\end{itemize}
	\item Mapping between physical and virtual addresses are managed by the
		operating system.
\end{itemize}

Advantages for virtual memory are:
\begin{itemize}
	\item Convenience --- all processes start at address 0.
	\item Easier to do paging.
	\item Protection and isolation of processes --- the kernel can make sure
		that processes can't access other processes memory, or the kernel's
		memory.
	\item Flexibility for the kernel to manage memory.
\end{itemize}

Virtual addresses are 32 bits, just like physical addresses.

What's in virtual address space?
\begin{itemize}
	\item text --- program code and read-only memory.
	\item data --- heap memory.
	\item stack --- starts at an address (\texttt{0x7fffffff}) and grows down.
\end{itemize}

\subsection{Address Translation}

\begin{itemize}
	\item Hardware provides a \emph{memory management unit (MMU)} which includes
		a \emph{relocation register}.
	\item In \emph{dynamic relocation}, the processes are loaded in a continuous
		block of memory. The OS puts the start address of the memory in the
		relocation register.
	\item The \emph{limit register} makes sure that the virtual address doesn't
		go over the amount of memory allocated for the process.
	\item There is more than one address translation per instruction. One for
		translating the address of the instruction, and any more for any
		addresses the instruction needs. This means that the OS can't be too
		involved, and this is why we have the MMU, which is fast because it's
		hardware.
	\item \emph{External fragmentation} can happen in dynamic relocation. If we
		don't have enough continuous space in physical memory, then we have a
		problem. There are heuristics for avoiding this. These are:
		\begin{itemize}
			\item First fit --- take the first section that fits
			\item Best fit --- take the smallest continuous section possible
			\item Worst fit -- take the largest continuous section possible
		\end{itemize}<++>
\end{itemize}<++>

\end{document}
% vim: tw=80
