\documentclass[12pt]{article}
\usepackage{geometry}
\usepackage{amsmath}
\usepackage{amsthm}
\usepackage{amssymb}
\usepackage{mathrsfs}
\usepackage{parskip}
\usepackage{enumerate}
\usepackage{stmaryrd}
\usepackage{listings}
\usepackage{fullpage}

\begin{document}

\title{CS 350 Notes}
\author{Matthew Visser}
\date{Sep 15, 2011}
\maketitle

\newcommand{\ie}{\textit{i.e.}}
\section{Program Execution}

\begin{itemize}
    \item Registers
        \begin{itemize}
            \item  Has program counter, stack pointer, data for operations
        \end{itemize}
    \item Memory
        \begin{itemize}
            \item Stack frame: contains activation records, control flow, state
                of program.
                \begin{enumerate}
                    \item Local variables
                    \item Return address
                    \item Function parameters
                \end{enumerate}
            \item Program code
            \item program data
        \end{itemize}
    \item CPU
\end{itemize}

\section{MIPS Register Usage}

See \texttt{kern/arch/mips/include/asmdefs.h} and slides \texttt{threads.pdf}
page 2-3.

\section{Threads}

\begin{itemize}
    \item Represents the control state of a program
    \item Has an associated \textit{context} which consists of:
        \begin{itemize}
            \item CPU state, including program counter, stack pointer,
                registers, execution mode (privilege)
            \item a stack, which is located in address space of thread's process
        \end{itemize}
    \item In short, contains all information you would need to suspend a program
        and then resume it later i.e.\ stack and cpu registers.
\end{itemize}

\subsection{Advantages}
\begin{itemize}
    \item Don't have to wait for a program that will take a long time, can
        switch between that and doing something useful.
    \item Take advantage of multi CPU
\end{itemize}

\subsection{Definitions}
\begin{description}
    \item[Thread Library:] Code that manages threads.
    \item[Thread Control Block:] Data structure that stores thread context.
\end{description}

\subsection{Notes}

\begin{itemize}
    \item OS/161 stores register values on stack, just stores stack pointer.
\end{itemize}

\subsection{Context Switching}

\begin{description}
    \item[Context Switching:] The act of pausing one thread, and resuming or
        starting another.
\end{description}

What happens during this?
\begin{enumerate}
    \item Save context
    \item Decide what runs next (Scheduler)
    \item Restore context of thread to run next \ie\ 
        \textit{Dispatching} a new thread.
\end{enumerate}

This is
\begin{itemize}
    \item Architecture specific
    \item Thread must save/restore carefully, since context changes often
    \item Can be tricky to understand because it's hard to define what point a
        thread stops in, and when it resumes.
\end{itemize}

When the context switch happens, the dispatching function returns to the yield
function of the \textit{new} thread, not the old one.

\end{document}
% vim: tw=80
