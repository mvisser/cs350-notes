\documentclass[12pt]{article}
\usepackage{geometry}
\usepackage{amsmath}
\usepackage{amsthm}
\usepackage{amssymb}
\usepackage{mathrsfs}
\usepackage{parskip}
\usepackage{enumerate}
\usepackage{stmaryrd}
\usepackage{listings}
\usepackage{fullpage}

\begin{document}

\title{CS 350 Notes}
\author{Matthew Visser}
\date{Nov  1, 2011}
\maketitle

\section{Exploiting Secondary Storage}

We have memory and we also have a hard disk that we can utilize.  We can write
memory pages to disk instead of RAM to increase our virtual memory size. This is
called \emph{paging}.

We utilize a \emph{two-level page table} to keep track of where the memory is.
See slide 46 of \texttt{vm.pdf}.

We can also use an \emph{inverted page table}.
\begin{itemize}
	\item We have one inverted table per frame of physical memory. 
	\item Because page tables are expensive, we don't want to go to them for
		every address translation. 
	\item If we have a TLB miss, we will first look in the inverted page table.
	\item It is highly likely that the memory we want is already in memory.
\end{itemize}

Once we have paging set up, how do we determine which pages we page in and out?
We could use a
\begin{itemize}
	\item FIFO approach --- pages that were loaded first get un-loaded first.
	\item Optimal --- predict the future and keep the pages we need
		(unrealistic, but would be nice).
	\item Random replacement --- just pick a random one.
\end{itemize}
The \emph{page reference string} tells us what pages we have.

\end{document}
% vim: tw=80
