\documentclass[12pt]{article}
\usepackage{geometry}
\usepackage{amsmath}
\usepackage{amsthm}
\usepackage{amssymb}
\usepackage{mathrsfs}
\usepackage{parskip}
\usepackage{enumerate}
\usepackage{stmaryrd}
\usepackage{listings}
\usepackage{fullpage}

\begin{document}

\title{CS 350 Notes}
\author{Matthew Visser}
\date{Nov 22, 2011}
\maketitle

\section{Devices and Device Controllers}

\begin{itemize}
	\item Examples
		\begin{itemize}
			\item Network interface
			\item graphics card
			\item storage (disk, tape)
			\item serial (mouse, keyboard)
			\item sound
			\item co-processors
		\end{itemize}

	\item How does the kernel talk to these devices? It talks to it through a
		\emph{device driver}.

	\item The driver talks to the device through a BUS.
		\begin{itemize}
			\item there are many kinds of BUS's.
			\item PCI, ISA
		\end{itemize}
	\item A \emph{controller} has registers tha tit uses to interact with the
		kernel:
		\begin{itemize}
			\item command --- how the kernel tells the device to do something
			\item status --- the device writes to these and the kernel reads
				from it so the device can tell the kernel what it's doing
			\item data --- data that the kernel can send to the device
		\end{itemize}
	\item A device uses interrupts to tell the kernel that it's done a task, and
		the kernel handles the interrupt to find out what to do next.
	\item Kernel has to copy memory from the device back to memory and vice
		versa, which takes a lot of CPU
	\item Direct Memory Access (DMA) prevents the writing, and the controler is
		smart enough to write its data to kernel memory directly without CPU
		involvement.
	\item The advantage of DMA is that the CPU isnt' busy with every device
		acces. The disadvantage is that the hardware is now much more complex,
		since the hardware must write to memory.
\end{itemize}

\end{document}
% vim: tw=80
