\documentclass[12pt]{article}
\usepackage{geometry}
\usepackage{amsmath}
\usepackage{amsthm}
\usepackage{amssymb}
\usepackage{mathrsfs}
\usepackage{parskip}
\usepackage{enumerate}
\usepackage{stmaryrd}
\usepackage{listings}
\usepackage{qtree}

\begin{document}

\title{CS 350 Notes}
\author{Matthew Visser}
\date{Oct 20, 2011}
\maketitle

\section{Assignment 2}

First things to do:
\begin{itemize}
	\item Write to console.
	\item Get \texttt{\_exit} working.
\end{itemize}

\subsection{\texttt{\_exit}}

\begin{itemize}
	\item Clean up process
		\begin{itemize}
			\item thread
			\item Address space
			\item more stuff as assignment progresses like process numbers, FDs,
				\texttt{etc}.
		\end{itemize}
	\item Start by looking at \texttt{thread\_exit}.
\end{itemize}

\subsection{Implement Argument Passing}

We need programs main to get the arguments:
\begin{verbatim}
int main (int argc, char **argv);
\end{verbatim}

\textbf{Note}: look at the hints page. The command to run a program and give it
arguments in os/161 is the \texttt{p command arg1 arg2} command.

Issuing the \texttt{p} command calls the function in the kernel
\texttt{cmd\_progthread()} which calls the function
\texttt{run\_program()}.

For \texttt{cmd\_progthread}, it has two arguments. The first pointer is to
\texttt{argv}. Make sure \texttt{argv} has a \texttt{NULL} at the end of it.
\texttt{argc} is the count of the number of arguments (or the last index in
\texttt{argv}.

We can place \texttt{argv}:
\begin{itemize}
	\item On the stack
	\item In its own segment
	\item somewhere crazy (as long as it works)
\end{itemize}
Look at \texttt{dumbvm} and see how it creates a stack, especially if you're
putting arguments on the stack.

After copying \texttt{argv} and \texttt{argc}, we have to make the program aware
of them by putting the arguments in \texttt{a0,a1}. To see how to do this, look
at \texttt{run\_program()}, and \texttt{md\_usermode()}. The second function is
the one that switches from kernel-mode to user-mode.

\subsection{System Calls}

\begin{itemize}
	\item Files
		\begin{itemize}
			\item \texttt{open}
			\item \texttt{close}
			\item \texttt{write}
		\end{itemize}
	\item Processes
		\begin{itemize}
			\item \texttt{fork}
			\item \texttt{waitpid}
			\item \texttt{\_exit}
			\item \texttt{execv}
		\end{itemize}
	\item Exceptions
\end{itemize}

Do Files (without \texttt{execv} and processes first, in whatever order. Then do
exceptions and \texttt{execv}.

\subsubsection{File-Related System Calls}

\Tree
[
.{\texttt{vfs.h} --- Virtual File System}
{EMUFS --- root file system}
{SFS --- \texttt{DISK1.img}, \texttt{DISK2.img}}
{Console Device}
]

The File System Interface:
\begin{itemize}
	\item \texttt{vfs\_open( path, flags, vnode )}
		\begin{itemize}
			\item \texttt{path} is the path to the file
			\item \texttt{flags} are the read-write flags
			\item \texttt{vnode} is returned, and is an abstraction of an open
				file.
		\end{itemize}
	\item \texttt{vfs\_close( vnode )} --- closes a file
\end{itemize}

Look at the file \texttt{vnode.h} for file operations. Kernel-level file
operations are done for you, but the user level file operations must be
implemented. Look at slides 1 to 7 of the file system notes.

\subsubsection{Process System Calls}

Need to write \texttt{md\_forkentry}.

\section{Address Space}

\subsection{Loading a Program into Address Space --- ELF Files}

\begin{itemize}
	\item Format for executables is called ELF. It is used by the kernel to
		create the address space for a process.
	\item And ELF file contains
		\begin{itemize}
			\item An \emph{ELF Header}
			\item Program headers
				\begin{itemize}
					\item Contains one header per segment.
					\item The header contains the following:
						\begin{itemize}
							\item Virtual address of segment
							\item Length of the segment
							\item Location of the start of the image in ELF file
							\item Length of the image in ELF file.
						\end{itemize}
				\end{itemize}
			\item \emph{Segments} $1\dots n$.
			\item \emph{Section Headers}
		\end{itemize}
	\item A \emph{segment} is a chunk of data to be loaded into the address
		space.
	\item Kernel must fill rest of segment with 0's.
	\item OS/161 expects the ELF file to have two segments:
		\begin{itemize}
			\item Program code
			\item Data
		\end{itemize}
\end{itemize}

\subsection{ELF Sections and Segments}

\begin{itemize}
	\item The ELF file contains different sections
		\begin{itemize}
			\item \texttt{.text} --- program code
			\item \texttt{.rodata} --- read-only global data
			\item \texttt{.data} --- initialized global data
			\item \texttt{.bss} --- uninitialized global data
			\item \texttt{.sbss} --- small uninitialized global data
		\end{itemize}
	\item Note all of these are present in every ELF file.
	\item \texttt{.text} and \texttt{.rodata} for the \texttt{text} segment. The
		rest for the \texttt{data} segment.
\end{itemize}<++>


\end{document}
% vim: tw=80
