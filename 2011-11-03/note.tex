\documentclass[12pt]{article}
\usepackage{geometry}
\usepackage{amsmath}
\usepackage{amsthm}
\usepackage{amssymb}
\usepackage{mathrsfs}
\usepackage{parskip}
\usepackage{enumerate}
\usepackage{stmaryrd}
\usepackage{listings}
\usepackage{fullpage}

\newtheorem{defn}{Definition}[subsection]

\begin{document}

\title{CS 350 Notes}
\author{Matthew Visser}
\date{Nov  3, 2011}
\maketitle

\section{Paging}

\subsection{Page Replacement}

\begin{defn}
	\emph{Locality} is a property of the page reference string.  
	\begin{description}
		\item[Temporal] pages that are used will likely be used again.
		\item[Spacial] pages ``close'' to those that have been used will
			probably be used.
	\end{description}
\end{defn}

\begin{itemize}
	\item A page that has been recently used will probably be used in the near
		future.

	\item Frequency-based page replacement
		\begin{itemize}
			\item Look at the pages that are used the most, don't remove them
				and remove pages used less.
			\item Not practical since the MMU would need to increment some
				counter, which it can't do.
			\item Old references are never forgotten
			\item New pages may be paged out soon after since the counts for old
				pages are higher
		\end{itemize}
	\item Least Recently Used
		\begin{itemize}
			\item Based on principle of temporal locality. Replace page that
				hasn't been used in a long time
			\item TO implement, it is necessary to track each page's recency.
			\item Although LRU and variants have many applications, it is not
				considered practical for virtual memory. (This is because the
				MMU still needs to keep track of what was used.) It is
				inefficient to keep track of LRU.
		\end{itemize}
	\item The \emph{use bit} or \emph{reference bit} is a bit found in the TLB
		\begin{itemize}
			\item It is set by the MMU each time the page is used
			\item Can be read and modified by OS
			\item OS copies information into page table
		\end{itemize}
	\item What if the MMU doesn't have a use bit?
		\begin{itemize}
			\item The kernel can emulate the bit at the cost of extra
				exceptions.
				\begin{enumerate}
					\item When a page is loaded, mark it as invalid.
					\item If a program attempts to access the page, an exception
						will occur.
					\item In its exception handler, the OS sets the page's
						simulated use bit to true, and marks the page valid.
				\end{enumerate}
			\item This technique requires that the OS maintain extra information
				for each page.
		\end{itemize}
	\item Clock Replacement Algorithm
		\begin{itemize}
			\item The clock algorithm is one of the simplest algorithms to
				exploit the use bit
			\item Identical to FIFO except that a page is skpped if its use bit
				is set.
			\item Can be visualized as a victim pointer that cycles through the
				page frames. The pointer moves whenever a replacement is
				necessary.
		\end{itemize}
	\item Enchanced Second Chance Algorithm
		\begin{itemize}
			\item prefer to unload pages that are not recently used and not
				modified.
			\item If there are no such pages, look for not recently used but
				modified.
			\item If there are no such pages, another run through the algorithm
				will find one because we reset the use bits as we're going
				through.
		\end{itemize}
\end{itemize}

\subsection{Page Cleaning}

\begin{defn}{Cleaning}

	Cleaning a page means copying the page to secondary storage.

\end{defn}

\begin{description}
	\item[Synchronous] happens at the time the page is replaced, during fault
		handling. Page is first cleaned by copying it to secondary storage then
		a new page is brought in to replace it.
	\item[Asynchronous] happens before a page is replaced, so that page fault
		handling can be faster.
		\begin{itemize}
			\item Asynchronous cleaning may be implemented by dedicated OS page
				cleaning threads that sweep through in-memory pages cleaning
				modified pages.
		\end{itemize}
\end{description}

\end{document}
% vim: tw=80
