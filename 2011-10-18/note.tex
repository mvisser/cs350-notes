\documentclass[12pt]{article}
\usepackage{geometry}
\usepackage{amsmath}
\usepackage{amsthm}
\usepackage{amssymb}
\usepackage{mathrsfs}
\usepackage{parskip}
\usepackage{enumerate}
\usepackage{stmaryrd}
\usepackage{listings}
\usepackage{fullpage}

\begin{document}

\title{CS 350 Notes}
\author{Matthew Visser}
\date{Oct 18, 2011}
\maketitle

Mistake from last class: the volatile should go by the pointer.

\section{Shared Virtual Memory}

Shared object files are files like \texttt{*.so} or \texttt{*.dll}. These are
shared libraries that are loaded into memory.

\subsection{Kernel Address Space}

Having the kernel use virtual addresses is confusing, because the kernel needs
the virtual address mapping to work before it can use it. Because of this, we
just have the kernel use physical addresses.

The kernel can have its own page table, but it's tricky to do so. If the kernel
gets an address, it's physical, so it must reverse the mapping first.

Linux and most other OS's have a shared segment for every process that holds an
address space in the kernel. MIPS helps the OS do this by reserving the upper
half of address space for the kernel. The upper half is divided into three
kernel segments: \texttt{kseg0}, \texttt{kseg1}, \texttt{kseg2}.

The MIPS processor helps by
\begin{enumerate}
	\item Addresses $\ge 0x80000000$ are \emph{privileged}.
	\item \texttt{kseg0}: virtual address \texttt{0x80000000} is mapped to
		physical address \texttt{0x00000000} without a TLB.
		\begin{itemize}
			\item This makes booting the machine a lot easier. We now know
				exactly what address the machine starts executing at, so we can
				put our boot loader there.
		\end{itemize}
	\item \texttt{kseg1} starts at \texttt{0xa0000000} and is mapped to
		\texttt{0x00000000} without a TLB and bypassing CPU cache. We have this
		segment for IO devices.
	\item \texttt{kseg2} is there for any addresses in \texttt{kseg0} that are
		over the address range. This is not used in OS/161.
\end{enumerate}

\textbf{Midterm up to slide 28 of \texttt{vm.pdf}}.

\section{Notes About Midterm}

\begin{tabular}{lp{400pt}}
	Date: & Oct 25\\
	Time: & 4:30 - 6:30\\
	Notes: & There is no lecture on this date. Instead, office hours are held
	from 1000h to 1100h in DC 3349, not in the lab.\\
\end{tabular}

How to study for the midterm?

There are some sample midterms on the course web page.

\section{Assignment 2 Discussion}

User programs to test with are in \texttt{testbin}. Different programs use
different system calls.

\subsection{Write and Exit}

Get write to console and \texttt{\_exit} working and run palin.
\begin{enumerate}
	\item Define a handler function and call it from
		\texttt{mips\_syscall}
	\item Copy data from user space.
	\item Do Logic of \texttt{write} or \texttt{\_exit}.
\end{enumerate}

How do we print data to the console? We use \texttt{kprintf}.
\begin{enumerate}
	\item \texttt{kprintf} uses a lock to write strings atomically.
	\item Calls \texttt{putch()} which puts one character to the console. The
		name of the console is called \texttt{/dev/generic/console} or
		``\texttt{con:}''. This device is \emph{asynchronous}.
\end{enumerate}

How do we copy data from user space into the kernel?

A kernel can dereference a pointer from user space because the kernel is mapped
into the address space of every process. We need to copy data because
\texttt{kprintf} could cause our thread to sleep. We might also want data to be
aligned in the kernel.

We can copy using \texttt{copyin/copyout} and \texttt{copyinstr/copyoutstr}.
These copy functions call \texttt{memcpy}. The string copy functions call
\texttt{copystr} which is like \texttt{strncpy}.

UIO Move $\to$ moves data from/ to kernel buffer and user level buffer.
Specified in an uio struct.

\end{document}
% vim: tw=80
