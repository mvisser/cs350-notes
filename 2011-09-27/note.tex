\documentclass[12pt]{article}
\usepackage{geometry}
\usepackage{amsmath}
\usepackage{amsthm}
\usepackage{amssymb}
\usepackage{mathrsfs}
\usepackage{parskip}
\usepackage{enumerate}
\usepackage{stmaryrd}
\usepackage{listings}
\usepackage{fullpage}

\begin{document}

\title{CS 350 Notes}
\author{Matthew Visser}
\date{Sep 27, 2011}
\maketitle

\section{Semaphores}

\subsection{The Producer-Consumer Problem}

Producers add items (to a list) and consumers consume them.

Need to be careful how you apply semaphores, if you reverse an order or don't
think about them, you could end up with thread blocking.

You also don't want consumers continually creating items. You can limit this
with a semaphore in the other direction.

There are many types of synchronization problems:
\begin{itemize}
    \item Dining philosopher
    \item Sleeping barber
    \item Cigarette smoker
    \item Readers and Writers
\end{itemize}

\section{Locks}

Locks are used to enforce mutual exclusion. You can use them in OS/161 by
calling \texttt{lock\_aquire()} and \texttt{lock\_release()}.

The implementation needs to enforce that the thread that acquires the lock is
the thread that releases the lock.

\section{Condition Variables}

A thread can call three operations
\begin{description}
    \item[wait:] Causes a thread to wait for a condition. Releases the lock it
        has on the critical section. The lock is re-acquired before returning
        from the wait procedure.
    \item[signal] If threads are blocked, one of them will be unblocked.
    \item[broadcast]  If threads are blocked, all of them are unblocked.
\end{description}

There are two semantics of using condition variables:
\begin{description}
    \item[Mesa Semantics:] Signalling thread stays in its CS. Waiting thread
        must wait for signalling thread and possibly other threads after
        unblocked.
    \item[Hoare Semantics:] Signalling thread leaves CS and waiting thread
        atomically enters its CS.
\end{description}

Semaphores accumulate calls to \texttt{V()}. Condition variables do not
accumulate, they either broadcast or signal.

\section{Monitors}

You can simulate having monitors by using a design pattern and code conventions,
listed on slide 30 of \texttt{synch.pdf}.

\section{Dead Locks}

Dead locks can occur when you have multi-threaded code. This type of problem
happens when two threads are permanently waiting for each other. This usually
happens when a thread holds one resource, and asks for another.

\textbf{Desirable properties in multi-threaded programs are:}
\begin{itemize}
    \item Safety -- threads are properly synchronized.
    \item Liveness -- threads all make progress (eventually). There are three
        violations to this:
        \begin{itemize}
            \item Starvation
            \item Dead Lock
            \item Live Lock
        \end{itemize}
        Having a cycle in resource allocation is a necessary condition for a
        dead lock, but it isn't sufficient.
\end{itemize}

\subsection{Dead Lock Prevention}

\begin{description}
    \item[No Hold and Wait:] Only request resources one at a time, if requesting
        multiple, request at once.
    \item[Preemption:]  Take resources away from a thread and give them to
        another.
    \item[Resource Ordering:] Assign each resource a number. Always request them
        in increasing order, \textit{i.e.}, always request lock A before lock B,
        no matter what thread or function.
\end{description}

\subsection{Dead Lock Recovery}

Need to detect there is a deadlock, then terminate one thread causing the lock.




\end{document}
% vim: tw=80
