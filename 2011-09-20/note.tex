\documentclass[12pt]{article}
\usepackage{geometry}
\usepackage{amsmath}
\usepackage{amsthm}
\usepackage{amssymb}
\usepackage{mathrsfs}
\usepackage{parskip}
\usepackage{enumerate}
\usepackage{stmaryrd}
\usepackage{listings}
\usepackage{fullpage}

\begin{document}

\title{CS 350 Notes}
\author{Matthew Visser}
\date{Sep 20, 2011}
\maketitle

\section{Threads}

\subsection{Thread Interface}

\begin{itemize} 
    \item Fork -- create new
    \item Exit -- destroy
    \item Yield -- let another thread run, but reschedule at some point.
    \item Sleep -- wait for an event
    \item Wake-up -- make threads sleeping on an address wake up.
\end{itemize}

\subsection{Fork}

Forking creates a new thread. It gives
\begin{itemize}
    \item A name -- for debugging
    \item a function to execute -- has 2 arguments
\end{itemize}

This function needs to 
\begin{enumerate}
    \item Allocate a stack
    \item Zero out the stack, store a pointer to the thread function on the
        stack's thread context. Stored at address \texttt{40(sp)}.
    \item Add thread to thread library. Create a \texttt{struct thread} for the
        new thread in the thread library and store \texttt{sp} in the thread
        structure.
    \item Extra:
        \begin{enumerate}
            \item Pass parameters to thread function
            \item Call \texttt{thread\_exit} when thread is done.
        \end{enumerate}
\end{enumerate}

\subsection{Scheduling Threads}

\begin{itemize}
    \item FIFO -- maintains a queue of threads that are ready to be scheduled.
        When a thread yields or forks, it's moved to the back of the queue.
        Queue is stored in kernel data segment in library.
\end{itemize}

\subsubsection{Preemption}

\begin{itemize}
    \item \texttt{Yield()} allows programs to voluntarily pause
    \item Sometimes it is desirable to make a thread stop running even if it
        has not called \texttt{Yield()}.
    \item an involuntary context switch is called \textit{preemption} of the
        running thread.
    \item To implement preemption, the thread library must have a means of
        taking over control, even though the application has not called a thread
        library function
    \item This form of taking control is normally called \textit{interrupts}.
    \item Easier to write code for a preemptive scheduler -- don't have to worry
        about yielding CPU. Makes the job of a programmer simpler.
\end{itemize}

\subsubsection{Interrupts}

\begin{itemize}
    \item An event that occurs while a program executes
    \item caused by system devices: timer, disk controller, network interface
    \item When an interrupt occurs, hardware automatically transfers control to
        a fixed location
    \item At the memory location, the thread library places a procedure called
        an \textit{interrupt handler}
    \item The handler normally
        \begin{enumerate}
            \item Saves current thread context. This is saved on the stack of
                the \textit{current} thread.
            \item determines what device caused interrupt and performs specific
                processing
            \item restores saved thread context and resumes execution
        \end{enumerate}
\end{itemize}

\subsubsection{Round-Robin Scheduling}

\begin{itemize}
    \item One example of preemptive scheduling.
    \item Similar to FIFO, but is preemptive
    \item Gives each thread a fixed time to execute before it is preempted.
    \item The time it executes is called the \textit{scheduling quantum}.
\end{itemize}

To implement:
\begin{itemize}
    \item Suppose the timer generates an interrupt at $t$ time units
    \item suppose the thread library wants a quantum $q=500t$
    \item To implement this, the thread library can maintain a variable called
        \texttt{running\_time} to track how long a thread has been running
        \begin{itemize}
            \item Reset to 0 when thread dispatched.
            \item When an interrupt occurs, the timer-specific code can
                increment \texttt{running\_time} and test its value, invoking
                \texttt{Yield()} if the time is equal to $q$.
        \end{itemize}
    \item See slide 23 of \texttt{threads.pdf} to see a diagram of stack.
\end{itemize}

\section{Synchronization}

\subsection{Concurrency} 

\begin{itemize}
    \item Threads run concurrently, share access to data, can run at the same
        time.
    \item Need to enforce \textit{mutual exclusion}.
    \item The part of the program where a shared object is accessed is called
        the \textit{critical section}.
\end{itemize}

\subsubsection{Mutual Exclusion}

\begin{itemize}
    \item Can ensure only one thread executes a critical section at a time.
    \item Server techniques to enforce
        \begin{itemize}
            \item Exploit special hardware specific instructions
            \item Use algorithms that rely on atomic loads and stores
            \item Control interrupts to ensure threads are not preempted while
                they are executing a critical section
        \end{itemize}
\end{itemize}

\subsubsection{Disabling interrupts}

On a uni-processor, only one thread is running at a time, so we can just disable
interrupts.

If the running thread enters a critical section, mutex can be violated if
\begin{enumerate}
    \item The thread is preempted
    \item The scheduler chooses a different thread to run
\end{enumerate}

Since preemption is caused by timer interrupts, mutex can beenforced by
disabling interrupts before a thread enters a critical section, and re-enabling
them when the thread leaves the critical section.

This is how OS/161 does this. It calls \texttt{splhigh()}.

Pros:
\begin{itemize}
    \item Doesn't require hardware-specific synchronization instructions
    \item Works for any number of threads
\end{itemize}

Cons:
\begin{itemize}
    \item Indistriminate: prevents all preemption.
    \item Ignoring timer interrupts has side effects.
    \item Will not enforce mutex on multi-processor.
\end{itemize}

\subsubsection{Priority Levels in MIPS}

Has levels from $0 \to 15$. From 0 to 14, interrupts are enabled.
\texttt{splhigh()} means to disable interrupts completely. The level is 15, and
no other thread is higher priority.



\end{document}
% vim: tw=80
