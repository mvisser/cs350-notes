\documentclass[12pt]{article}
\usepackage{geometry}
\usepackage{amsmath}
\usepackage{amsthm}
\usepackage{amssymb}
\usepackage{mathrsfs}
\usepackage{parskip}
\usepackage{enumerate}
\usepackage{stmaryrd}
\usepackage{listings}
\usepackage{fullpage}

\begin{document}

\title{CS 350 Notes}
\author{Matthew Visser}
\date{Nov  8, 2011}
\maketitle

\section{Paging}

\subsection{Prefetching}

\begin{itemize}
	\item Prefetching means we move virtual pages before we need them
	\item Same or more amount of work, but decreases apparent latency
\end{itemize}

\subsection{Page Size}

\begin{itemize}
	\item Large page sizes have less paging.
	\item Larger pages have fragmentation and an increased chance of paging
		memory you don't need.
	\item Super pages allow you to change page size at runtime.
\end{itemize}

\subsection{How much memory does a process need?}

Working set: $W$, Resident Set $R$.
\[ W \subseteq R \]

The working set is the heavily used portion of the program's address space.

The program's resident set is the set of pages in memory.

We define $W(s,\Delta)$ to be the set of pages referenced by a process during a
time interval $(t-\Delta,t)$. This is the working set of time $t$.

\subsection{Thrashing and Load Control}

We need a good level of the number of processes:
\begin{itemize}
	\item Too low and we idle resources
	\item Too high and we have too few resources per process.
	\item A system spending too much time is said to be \emph{thrashing.}
	\item Thrashing can be cured by shedding load:
		\begin{itemize}
			\item killing processes
			\item Suspending and swaping out processes.
		\end{itemize}
	\item Performance drops \emph{very} quickly when it starts thrashing.
	\item What processes do we suspend?
		\begin{itemize}
			\item Low priority
			\item blocked
			\item Large (frees lots of resources)
		\end{itemize}
\end{itemize}

\section{Copy-On-Write}

The reason to have this is that often, you want to duplicate an address space,
\textit{e.g.} \texttt{fork()}.

The way it works is it doesn't copy pages at first, but when one page tries to
write, then it copies.

This is useful for filesystem snapshots as well.

\section{Program Execution}

A running thread can be modeled as CPU bursts and IO bursts.

Threads are scheduled when:
\begin{itemize}
	\item They yield
	\item they are pre-empted after their quantum
\end{itemize}

Properties we want:
\begin{itemize}
	\item Enforce priority
	\item prevent starvation
	\item be fair
	\item minimize wait time
	\item maxmize throughput, better (minimal) turn-around time.
	\item Enforce CPU quota
\end{itemize}

There are two basic strategies:
\begin{itemize}
	\item FCFS --- a FIFO redy queue, non-preemptive
	\item Round Robin
	\item SHortest Job First
		\begin{itemize}
			\item Non-preemptive
			\item ready threads are scheduled according to the length of their
				next CPU burst.
			\item Requires knowledge of burst length:
				\[B_{i+1} = \alpha b_i + (1-\alpha)B_i\]
				where $B_i$ is the predicted length of the CPU burst and
				$b_i$ is the actual length. $0 \le \alpha \le 1$.
		\end{itemize}
\end{itemize}


\end{document}
% vim: tw=80
